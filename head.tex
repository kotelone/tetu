\documentclass[a4paper, 12pt]{article}
		
	\usepackage{mathtools}
	\usepackage{amssymb}
	\usepackage{amsmath}
	\usepackage{setspace}
	\usepackage{mathtext}
	\usepackage{listings}
	\usepackage{indentfirst}

	\usepackage{color}
	\usepackage{graphicx}
	%\usepackage{showframe}
	\usepackage{fancyhdr}
	\usepackage{wrapfig}    
	\usepackage{floatrow}
	
	% Параметры шрифта
	\usepackage[T1,T2A]{fontenc}
	\usepackage[utf8]{inputenc}
	\usepackage[english, russian]{babel}
	\frenchspacing
	\onehalfspacing

	% стилевой файл для оформления списка литературы по ГОСТ
	\bibliographystyle{utf8gost780u}
	
	%Настройка подписей к таблицам
	\floatsetup[table]{capposition = top}

	% Параметры страницы
	\hoffset=0mm            % Отступ от левого края + 1 дюйм
	\voffset=0mm            % Отступ от верхнего края + 1 дюйм
	\oddsidemargin=-5.4mm   % левое поле 25.4 - 5.4 = 20 мм
	\topmargin=-15.4mm      % верхнее поле 25,4 - 15,4 = 10 мм
	\textwidth=170mm        % ширина текста
	\textheight=260mm       % высота текста 297 (A4) - 40
	\topmargin=-15.4mm      % верхнее поле 25,4 - 15,4 = 10 мм
	\headheight=2mm         % место для колонтитула
	\headsep=5mm            % отступ после колонтитула
	\footskip=7mm           % отступ до нижнего колонтитула
	
	% Настройки листингов
	\lstset{ 
		language=Haskell,               % Язык программирования 
		numbers=left,                   % С какой стороны нумеровать
		%numberstyle=tinycolor{gray},   % Стиль нумерации строк
		stepnumber=2,                   % Шаг между линиями.
		numbersep=10pt,                  
		%backgroundcolor=color{white},  % Цвет подложки. 
		showspaces=false,               
		showstringspaces=false,         
		showtabs=false,                
		frame=single,                   % Добавить рамку
		%rulecolor=color{black},        
		tabsize=2,                      % Tab - 2 пробела
		breaklines=true,                % Автоматический перенос строк
		breakatwhitespace=true,         % Переносить строки по словам
		%title=lstname,                 % Показать название подгружаемого файла
		%keywordstyle=color{blue},      % Стиль ключевых слов
		%commentstyle=color{dkgreen},   % Стиль комментариев
		%stringstyle=color{mauve}       % Стиль литералов
	}
		
\begin{document} 

\newcommand{\nothing}		{}

\newcommand{\department}	{} 
\newcommand{\discipline}	{}
\newcommand{\lecturer}		{}

\newcommand{\labnumber}		{}
\newcommand{\labname}		{}

\input{./../config}
 
\begin{center}
	\textbf{МИНОБРНАУКИ РОССИИ}\\
	\textbf{САНКТ-ПЕТЕРБУРГСКИЙ ГОСУДАРСТВЕННЫЙ ЭЛEКТРОТЕХНИЧЕСКИЙ УНИВЕРСИТЕТ}\\ 
	\textbf{«ЛЭТИ» ИМ. В.И. УЛЬЯНОВА (ЛЕНИНА)}\\
	\textbf{Кафедра \department}\\

	\vspace*{70mm}

	\textbf{ОТЧЁТ}\\
	\textbf{по лабораторной работе №\labnumber}\\
	\textbf{по дисциплине «\discipline»}\\
	\textbf{Тема: \labname}\\

	\begin{figure}[b]
		\centering
		\begin{tabular}{ccc} 
			Студент\ifx\companion\nothing \space \else ы \space \fi гр. 7291 & 
				\underline{\hspace{3cm}} &  Фирюлин Д. И. \\\\
			\ifx\companion\nothing \else & \underline{\hspace{3cm}} &  \companion \\\\ \fi
			Преподаватель & \underline{\hspace{3cm}} & \lecturer\\\\
		\end{tabular}
		
		\vspace*{10mm}
		Санкт-Петербург
		
		\vspace*{2mm}
		\the\year
		\vspace*{4mm}
	\end{figure}
	
\end{center}
\thispagestyle{empty}

\newpage
%\tableofcontents

% =========================== PUT YOUR WATER HERE =============================


%\newpage
%\addcontentsline{toc}{section}{Список литературы}
%\bibliography{biblio}     %% имя библиографической базы (bib-файла) 

\end{document}





























